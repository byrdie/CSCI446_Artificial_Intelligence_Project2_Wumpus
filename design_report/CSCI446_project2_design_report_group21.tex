%%%%%%%%%%%%%%%%%%%%%%%%%%%%%%%%%%%%%%%%%
% Journal Article
% LaTeX Template
% Version 1.3 (9/9/13)
%
% This template has been downloaded from:
% http://www.LaTeXTemplates.com
%
% Original author:
% Frits Wenneker (http://www.howtotex.com)
%
% License:
% CC BY-NC-SA 3.0 (http://creativecommons.org/licenses/by-nc-sa/3.0/)
%
%%%%%%%%%%%%%%%%%%%%%%%%%%%%%%%%%%%%%%%%%

%----------------------------------------------------------------------------------------
%	PACKAGES AND OTHER DOCUMENT CONFIGURATIONS
%----------------------------------------------------------------------------------------

\documentclass{article}

%\documentclass{aastex}  % version 5.0 or prior
%\usepackage{natbib}



\usepackage{graphicx}
\usepackage{lipsum} % Package to generate dummy text throughout this template
%\usepackage[sc]{mathpazo} % Use the Palatino font
\usepackage[T1]{fontenc} % Use 8-bit encoding that has 256 glyphs
\linespread{1.05} % Line spacing - Palatino needs more space between lines
\usepackage{microtype} % Slightly tweak font spacing for aesthetics

\usepackage[margin=1in,columnsep=20pt]{geometry} % Document margins
\usepackage{multicol} % Used for the two-column layout of the document
\usepackage[hang, small,labelfont=bf,up,textfont=it,up]{caption} % Custom captions under/above floats in tables or figures
\usepackage{booktabs} % Horizontal rules in tables
\usepackage{float} % Required for tables and figures in the multi-column environment - they need to be placed in specific locations with the [H] (e.g. \begin{table}[H])
\usepackage{hyperref} % For hyperlinks in the PDF
\usepackage{subcaption}

\usepackage{lettrine} % The lettrine is the first enlarged letter at the beginning of the text
\usepackage{paralist} % Used for the compactitem environment which makes bullet points with less space between them
\usepackage{amsmath}
\usepackage{abstract} % Allows abstract customization
\renewcommand{\abstractnamefont}{\normalfont\bfseries} % Set the "Abstract" text to bold
\renewcommand{\abstracttextfont}{\normalfont\small\itshape} % Set the abstract itself to small italic text

\usepackage{titlesec} % Allows customization of titles
%\renewcommand\thesection{\Roman{section}} % Roman numerals for the sections
%\renewcommand\thesubsection{\Roman{subsection}} % Roman numerals for subsections
%\renewcommand\thesubsubsection{\Alph{subsubsection}} % Roman numerals for subsections
\titleformat{\section}[block]{\LARGE\scshape}{\thesection}{1em}{} % Change the look of the section titles
\titleformat{\subsection}[block]{\Large\scshape}{\thesubsection}{1em}{} % Change the look of the section titles
\titleformat{\subsubsection}[block]{\large\scshape}{\thesubsubsection}{1em}{} % Change the look of the section titles

\usepackage{fancyhdr} % Headers and footers
\pagestyle{fancy} % All pages have headers and footers
\fancyhead{} % Blank out the default header
\fancyfoot{} % Blank out the default footer
\fancyhead[C]{Montana State University \quad $\bullet$ \quad CSCI 466 Artificial Intelligence \quad $\bullet$ \quad Group 21} % Custom header text
\fancyfoot[RO,LE]{\thepage} % Custom footer text

\newcommand{\ve}[1]{\boldsymbol{\mathbf{#1}}}

\title{\vspace{-15mm}\fontsize{24pt}{10pt}\selectfont\textbf{CSCI 446 Artificial Intelligence \\[2mm] Project 2 Design Report} } % Article title
\date{\today}
\author{
\large
\textsc{Roy Smart} \and \textsc{Nevin Leh} \and \textsc{Brian Marsh}\\[2mm] % Your name
}


%----------------------------------------------------------------------------------------

\begin{document}

	\maketitle % Insert title
	\thispagestyle{fancy} % All pages have headers and footers
	\normalsize

	\section{Introduction}
	
		\textit{Logic and the Wumpus World} is an artificial intelligence problem first proposed by Michael Genesereth, and described in detail by his student Stuart Russel\cite{ai}.  The problem involves navigating an environment known as the Wumpus World using logic to avoid dangers as the agent attempts to reach a goal.  The environment is a square grid of tiles that can be empty or contain gold for the goal state, a pit that the agent will fall into, or a monster known as a Wumpus.  As the agent navigates the environment, a score is calculated from the various actions that the agent makes.  The objective of the game is thus to maximize the score, which can be achieved by using logic to find the most efficient route.
	
	\section{Problem Statement}
	
		To solve the Wumpus World problem, we will develop a problem generator that creates the Wumpus worlds in which the agent will navigate.  We will then implement a logic system that uses unification and resolution on first-order rules that allow the agent to navigate.  The reasoning system will use the following information: the stench when near a Wumpus, the breeze when near a pit, the scream of a Wumpus as it is killed, and the wall when ran into.  Additionally, a reactive agent will be created that makes decisions based upon random decisions regarding which cells are safe and which are not.  We will test the logic system on Wumpus worlds of sizes {5x5, 10x10,…, 25x25}.  Our performance metrics are as follows: number of times the gold is found, number of Wumpus killed, number of times the explorer falls into a pit, number of times the Wumpus kills the explorer, and number of cells explored.
	
	\section{Software Design}
		Our design will incorporate three main classes: \texttt{World}, \texttt{EnvironmentEngine}, and \texttt{Agent}. 
		These classes have all of the important attributes and functions needed to implement this game.
		The \texttt{World} class is responsible for creating and maintaining worlds.
		The \texttt{EnvironmentEngine} is responsible for checking the move an agent selects and then telling the agent the result of the move.
		Finally, the \texttt{agent} is an abstract class that uses information provided by the \texttt{EnvironmentEngine} to decide what the next move is. The three implemented versions of \texttt{Agent} are \texttt{HumanAgent}, \texttt{ReactiveAgent} , and \texttt{ReasoningAgent}.
		
		The \texttt{World} is used to create both the fully populated ''Master World`` and blank ''Agent World``. 
		A world will consist of a vector of vectors that each hold an int.
		Each int represents one square in the world, with each bit representing the presence of a certain feature. 
		We chose to represent the board this way because it seemed more simple than using a whole object for each tile.
		We decided to use a vector of vectors because they are much more forgiving to work with than two dimensional arrays in c++.
		The \texttt{World} class will also be responsible for keeping some important data such as number of wumpi and pits for general usage throughout the program.
		
		The \texttt{EnvironmnetEngine} is responsible for holding the master world object telling the agents about the world as it is explored.
		It's main job is to operate a loop that constantly calls the \texttt{update\_world} method in the agent. 
		The idea is that the \texttt{update\_world} returns the agents next move. 
		The \texttt{EnvironmentEngine} then applies that move to the master world.
		The next time \texttt{update\_world} is called the \texttt{EnvironmentEngine} includes the results of the attempted move.
		These results include the agents new position, the integer that represents the facts about that square, and miscellaneous facts such as whether a bump was detected. 
		This system is a good option because the same \texttt{EnvironmnentEngine} can be used for all of the agents. 
		
		The final, and most important class, is the \texttt{Agent} class. We chose create an abstract class because 
		   
		
	
		The design of our software is centered around the environment engine. 
		This class servers to link all components of the program and drives the main processes involved with this project. 
		

		
		\subsection{Wumpus World Generation}
		
			The wumpus world will be represented 
		
		\subsection{Environment Engine}
		
		\subsection{Reasoning Agent}
		
			\subsubsection{Inference Engine}
			
				Our inference engine will use the function \textsc{PL-Resolution}($KB$, $\alpha$) described by Russel and Norvig in Figure 7.12 \cite{ai} to perform \textit{resolution}. 
				Resolution allows the inference engine to use a knowledge base of facts to ascertain new details about the wumpus world. 
				The function \textsc{PL-Resolution} accepts a knowledge base, $KB$ and $\alpha$, the query to be checked. Of course, \textsc{PL-Resolution} is based on propositional logic clauses, but according to Section 9.5.2 of Russel and Norvig, we can adapt it to operate using first order logic clauses if we we modify the function \textsc{PL-Resolve($C_i$, $C_j$)} to find two variables complementary if one \textit{unifies} with the negation of the other. 
				
				To check if two variables can be unified, we will use the function \textsc{Unify}($x$, $y$, $\theta$) provided in Figure 9.1 of Russel and Norvig.
				In the interest of efficiency, we will also implement a \textit{predicate indexing}, described in Section 9.2.3 of Russel and Norvig\cite{ai}.
				The predicate index provides \textsc{Unify}() an efficient way to only attempt unification on sentences that are possible to be unified. 
				We will accomplish this by constructing lists of all facts that use a certain predicate, e.g. all facts that contain the predicate $Adjacent(r,s)$ will be placed into a list.
				Then, if \textsc{Unify()} is asked to unify a sentence containing $Adjacent(u,v)$, it will only have to search the list described in the previous sentence instead of the entire knowledge base.
				
			\subsubsection{Knowledge Base}
			
			\subsubsection{Pathfinding}
			
		\subsection{Reactive Agent}
			
		
		
	\section{Experiment Design}
	
	To analyze the performance of our agents, we are asked to vary various parameters of the wumpus World. These parameters include: the size of the board, and the number of wumpi, pits, and barriers. 
	
	
	

	




	%\bibliographystyle{apj}
	\bibliographystyle{unsrt}	
	\bibliography{sources}
\end{document}
